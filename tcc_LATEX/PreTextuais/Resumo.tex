% ---
% RESUMOS
% ---

% resumo em português
\setlength{\absparsep}{18pt} % ajusta o espaçamento dos parágrafos do resumo
\begin{resumo}
O crescente aumento da demanda energética mundial, em um contexto onde a preocupação com a sustentabilidade ambiental é cada vez mais relevante, aponta para a necessidade de estudos acerca de geração de energia e calor a partir de fontes renováveis. Uma das fontes com maior potencial nesse contexto é a biomassa lignocelulósica, proveniente de resíduos de madeira, que pode ser densificada na forma de pellets para um transporte e armazenamento otimizado. No Brasil, esse tipo de biomassa apresenta potencial de crescimento, sendo necessários estudos sobre equipamentos e formas de conversão dessa biomassa em energia. O presente estudo busca caracterizar um queimador de biomassa multi-estágios alimentado com pellets em intervalos regulares de tempo, onde a biomassa inicialmente é volatilizada e em seguida oxidada no topo do dispositivo. O equipamento do presente trabalho replica um dispositivo comercial, usado para calor de processo em indústrias e para aquecimento predial. A caracterização inicialmente é feita qualitativamente, observando-se o atingimento de regime permanente e a intensidade de chama. Observou-se que para potências mais altas de operação o dispositivo não atingiu uma condição de regime permanente, havendo excesso de combustível não queimado no reator. Nesse teste qualitativo também observou-se uma altura de chama maior para menores potências, resultado das condições fluidodinâmicas proporcionadas pela geometria do reator. Na caracterização quantitativa foi possível determinar a temperatura média dos gases de combustão para as potências testadas, obtendo-se valores entre 540°C e 610°C. Observou-se um caráter transiente significativo na temperatura e na emissão de poluentes entre cada alimentação de combustível. A variação da concentração de cada poluente nesse intervalo de tempo foi condizente com o esperado por dados da literatura. A emissão média de monóxido de carbono encontra-se dentro dos níveis máximos estabelecidos pelo Conselho Nacional do Meio Ambiente (CONAMA) para reatores dessa potência.

 \textbf{Palavras-chave}: Biomassa. Pellets. Queimador. Poluentes. Energia Renovável.
\end{resumo}
