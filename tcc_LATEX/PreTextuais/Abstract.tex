% ---
% Abstract
% ---

% resumo em inglês
\begin{resumo}[Abstract]
 \begin{otherlanguage*}{english}
The growing increase in the world energy demand, in a context where the concern with environmental sustainability is increasingly relevant, leads to the necessity of studies in the field of energy and heat generation from renewable sources. One of the sources with the greatest potential in this context is lignocellulosic biomass, from wood residues, which can be densified in the form of pellets for optimal transport and storage. In Brazil, this type of biomass has growth potential, requiring studies on equipment and ways of converting this biomass into energy. The present study aims to characterize a multi-stage biomass burner fed with pellets at regular time intervals, where the biomass is initially volatilized and then oxidized at the top of the device. The equipment of this work replicates a commercial device, used for process heat in industries and for building heating. The characterization is initially done qualitatively, observing the achievement of steady state and the flame intensity. It was observed that for higher operating powers the device did not reach a steady state condition, with excess of unburned fuel in the reactor. In this qualitative test, a higher flame height for lower powers was also observed, as a result of the fluid dynamic conditions provided by the reactor geometry. In the quantitative characterization, it was possible to determine the average temperature of the combustion gases for the tested powers, obtaining values between 540°C and 610°C. A significant transient behavior was observed regarding the temperature and emission of pollutants between each fuel supply. The variation in the concentration of each pollutant in this time interval was consistent with what was expected from literature data. The average emission of carbon monoxide is within the maximum levels established by the Brazilian National Council for Environment (CONAMA) for reactors of this power.

   \textbf{Keywords}: Biomass. Pellets. Burner. Pollutants. Renewable Energy.
 \end{otherlanguage*}
\end{resumo}
